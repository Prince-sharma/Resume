%% Based on a TeXnicCenter-Template by Gyorgy SZEIDL.
%%%%%%%%%%%%%%%%%%%%%%%%%%%%%%%%%%%%%%%%%%%%%%%%%%%%%%%%%%%%%

%------------------------------------------------------------
%
\documentclass[11pt]{article}%
\renewcommand{\baselinestretch}{1} 
%Options -- Point size:  10pt (default), 11pt, 12pt
%        -- Paper size:  letterpaper (default), a4paper, a5paper, b5paper
%                        legalpaper, executivepaper
%        -- Orientation  (portrait is the default)
%                        landscape
%        -- Print size:  oneside (default), twoside
%        -- Quality      final(default), draft
%        -- Title page   notitlepage, titlepage(default)
%        -- Columns      onecolumn(default), twocolumn
%        -- Equation numbering (equation numbers on the right is the default)
%                        leqno
%        -- Displayed equations (centered is the default)
%                        fleqn (equations start at the same distance from the right side)
%        -- Open bibliography style (closed is the default)
%                        openbib
% For instance the command
%           \documentclass[a4paper,12pt,leqno]{article}
% ensures that the paper size is a4, the fonts are typeset at the size 12p
% and the equation numbers are on the left side
%
\usepackage{amsmath}%
\usepackage{hyphenat}%
\usepackage{amsfonts}%
\usepackage{amssymb}%
\usepackage{graphicx}
\usepackage{lipsum}
\usepackage{ragged2e}
\usepackage{bm}
\usepackage[margin = 20 mm]{geometry}
 \geometry{
 a4paper,
 total={210mm,297mm},
 left=20mm,
 right=20mm,
 top=20mm,
 bottom=10mm,
 }
\usepackage{enumerate}
\usepackage{hyperref}
%-------------------------------------------

\pagenumbering{gobble}
\begin{document}
\justifying
\noindent
\vspace*{3.5cm}
\begin{flushleft}
\bf{\bf{\LARGE{A}\Large{CHIEVEMENTS}}}
\end{flushleft}
\vspace{-6mm}
\hrulefill
\vspace{-2mm}
\flushleft {\bf{\large Scholastic}}
\vspace{-2mm}
\begin{itemize}
\setlength\itemsep{0.01em}
\item Secured All India Rank \textbf{457} in \textbf{JEE advanced} (0.2 million Aspirants) \hfill {{{\em {2017}}}}
\vspace{-1mm}
\item Achieved a \textbf{Percentile 99.88} in \textbf{JEE Mains} (1.2 million Aspirants) \hfill {{{\em {2017}}}}
\vspace{-1mm}
\item Among \textbf{Top 20} at State level in \textbf{NSEA} (Astronomy Olympiad) and selected for {\bf INAO} \hfill {{{\em {2016}}}}
\vspace{-1mm}
\end{itemize}
\vspace{-5mm}
\flushleft {\bf{\large Other}}
\vspace{-3mm}
\begin{itemize}
    \item {\bf 1\textsuperscript{st} place} in 10,000+ teams in {\bf TechFest IITB's National Coding Competiton} : {\bf Enigma}
    \vspace{-2.2mm}
    \item Won {\bf Scratch Day}, coding competition organized by Web and Coding Club, IIT Bombay
    \vspace{-2.2mm}
    \item {\bf National Winner} in {\bf  IPL Auction 2018} organized by Entrepreunership Cell, IIT Bombay
    \vspace{-0.22cm}
    \item Placed {\bf 3\textsuperscript{rd}} in annual {\bf Jigyasa: Science Quiz 2017} organized by {\bf University of Mumbai}
    \vspace{-2.2mm}
    \item \textbf 2 Time {\bf City Winner} in {\bf Vodafone Derek$'$s} {\bf Faster Smarter Better Challenge $'$14, $'$15}
\end{itemize}
\begin{flushleft}
\vspace{0mm}
\bf{\LARGE{K}\Large{EY} \LARGE{P}\Large{ROJECTS} \LARGE{U}\Large{NDERTAKEN}}
\end{flushleft}
\vspace{-7mm}
\hrulefill
\vspace{-3mm}

{\flushleft \bf \large{ChordIt - Chord Sequence Extraction using Machine Learning}} \hfill {{{\em {Summer '18}}}}\\
{\em{Institute Technical Summer Project \textbar Web and Coding Club, IIT Bombay}}
\begin{itemize}
\vspace{-0.2cm}
\setlength\itemsep{0.01em}
\item Read literature on the use of machine learning in {\bf Chord Sequence Extraction} problem
\vspace{-1.2mm}
\item Researched {\bf Convex Optimization Methods} and selected {\bf Stochastic Gradient Descent}
\vspace{-1.2mm}
\item  Processed the music data to extract {\bf $\mathbb{R}^{12}$ Pitch Class Profile} vectors using optimized routines
\vspace{-1.2mm}
\item Achieved {\bf 95\%} train accuracy and {\bf 86\%} test accuracy on {\bf 2000} chords of different instruments
\vspace{-1.2mm}
\item  Incorporated minibatches in SGD Optimizer for {\bf Out-of-Core Learning} on online data
\vspace{-1.2mm}
\end{itemize}
\vspace{-4mm}

{\flushleft \bf \large{Machine Learning Models | Course Project}} \hfill {{{\em{Autumn '18}}}} \\
{\em{Prof. Sunita Sarawagi \textbar Dept. of Computer Science \& Engineering, IIT Bombay}}
\vspace{-1mm}
\begin{itemize}
\setlength\itemsep{0.01em}
\vspace{-1mm}
\item Implemented a regression tree using {\bf Modified CART Algorithm} and tested the accuracy for
different heuristic loss functions on different datasets
\vspace{-1mm}
\item Incorporated various pruning criteria to maximize the accuracy on different loss functions.
\vspace{-1mm}
\item Classified data using {\bf Logistic, Hinge,} and {\bf Perception} loss functions and {\bf L\textsuperscript{2}, L\textsuperscript{4} Regualiser}
\end{itemize}
\vspace{-0.5cm}


{\flushleft \bf \large{Effect of Reviews \& Responses on Consumer Decision | Web Scraping}} \hfill {{{\em{Summer '18}}}} \\
{\em{Prof. Arti Kalro \textbar Shailesh J. Mehta School of Management, IIT Bombay}}
\vspace{-1mm}
\begin{itemize}
\setlength\itemsep{0.01em}
\vspace{-1mm}
\item {\bf Scraped Zomato} for a list of restaurants at different price points in Mumbai
\vspace{-1mm}
\item For each restaurants, scraped reviews, no. of likes, comments, no. of photos added, management responses using {\bf Selenium} for {\bf Automated Web Scraping}
\vspace{-1mm}
\item For each reviewer, extracted no. of reviews, followers, images posted, etc
\vspace{-1mm}
\item Used {\bf Dynamic Programming} techniques to reduce runtime of web scraper
\end{itemize}
\vspace{-0.5cm}

{\flushleft \bf \large{InstiApp | Open Source App Development}}   \hfill {{\em{Autumn '18} }}\\
{\em{IITB Developer Community}}
\begin{itemize}
\vspace{-2mm}
\setlength\itemsep{0.01em}
\item Part of a team of {\bf 10+ developers} in making an {\bf Open Source Android App} for the students, professors, student bodies and visitors of IIT Bombay
\vspace{-1.2mm}
\item Solved many {\bf UI/UX} as well as {\bf Core Bugs} and also implemented new features
\vspace{-1.2mm}
\item Actively involved in testing, reviewing new features and reporting bugs for the app
\vspace{-1.2mm}
\item The app currently has {\bf 5000+ downloads} and a {\bf 4.9 $\star$ rating} on Google Play
\end{itemize}
\vspace{-0.5cm}


\pagebreak
\begin{flushleft}
\vspace{0mm}
\bf{\LARGE{O}\Large{THER} \LARGE{P}\Large{ROJECTS}}
\end{flushleft}
\vspace{-7mm}
\hrulefill
\vspace{-3mm}

{\flushleft \bf \large{Competitive Coding | Reading Project}} \hfill {{{\em{Summer '18}}}} \\
{\em{WnCC Seasons of Code 2018, IIT Bombay}}
\vspace{-1mm}
\begin{itemize}
\setlength\itemsep{0.01em}
\vspace{-1mm}
\item Studied the essential {\bf Data Structures \& Algorithms} and their underlying theory 
\vspace{-1mm}
\item Researched different {\bf Programming Paradigms} and their uses in various problems
\vspace{-1mm}
\item Practically applied all the concepts in {\bf Competitive Coding} on various online judges
\vspace{-1mm}
\item Covered topics like Dynamic Programming, Graph Algorithms, BackTracking \& more
\end{itemize}
\vspace{-0.5cm}

{\flushleft \bf \large{Heart Rate Monitor | Course Project}} \hfill  \hfill {{\em{Autumn '18} }}\\
{\em{Prof. Siddharth Tallur \textbar Department of Electrical Engineering, IIT Bombay}}
\begin{itemize}
\vspace{-2mm}
\setlength\itemsep{0.01em}
\item Studied {\bf Photoplethysmograms} and designed a heart rate monitor using Integrated Circuits, and {\bf TCRT5000} IR-LED phototransistor pair which sensed the blood flow through a finger
\vspace{-1.2mm}
\item Analyzed the waveform on a DSO to measure systolic and diastolic heart rate.
\vspace{-1.2mm}
\end{itemize}
\vspace{-0.4cm}


{\flushleft \bf \large{Walking Stick for the Blind | Social Innovation}} \hfill {{{\em{Autmumn '17 - Spring 18}}}} \\
{\em{National Innovation Cell, National Service Scheme, IIT Bombay}}
\vspace{-2mm}
\begin{itemize}
\setlength\itemsep{0.01em}
\item Designed and manufactured a walking stick which detects obstacles using {\bf Ultrasonic Sensors}
\vspace{-1.2mm}
\item Programmed {\bf Arduino} to simultaneously handle {\bf 3 sensors} and soldered the ancillary circuit
\vspace{-1.2mm}
\item Researched the market and chose {\bf cost-effective components} to make the product affordable
\vspace{-1.2mm}
\item Prepared a {\bf Questionnaire} to understand the problems and challenges faced by the blind
\end{itemize}
\vspace{-0.5cm}


\begin{flushleft}
\bf{\LARGE{P}\Large{OSITIONS} \LARGE{O}\Large{F} \LARGE{R}\Large{ESPONSIBILITY}}
\vspace{-3mm}
\end{flushleft}
\vspace{-4mm}
\hrulefill
\vspace{-0.2cm}
{\flushleft \bf \large{Convener | Web and Coding Club}} \hfill {{{\em {Apr '18 - Present}}}}\\{\em{Institute Technical Council, IIT Bombay}}
\begin{itemize}
\setlength\itemsep{0.01em}
\vspace{-0.3cm}
\item Responsible for conducting boot-camps, events, competitions and managing the club\textsc{\char13}s resources with a 
long term goal of creating a thriving programming community in the institute
\vspace{-1.2mm}
\item Organized bootcamps on {\bf Python/Libraries}, {\bf Machine Learning}, {\bf Git \& GitHub}
\vspace{-1.2mm}
\item Ideated \& Conducted a {\bf Coding Crypt Hunt} for Freshmen via a self-made {\bf Chatbot}
\vspace{-1.2mm}
\item Organized an introductory session on {\bf Web Scraping} for Management Post-Grad Students
\end{itemize}
\vspace{-1mm}
\vspace{-5mm}
{\flushleft \bf \large{Coordinator, Competitions \& LYP | Mood Indigo, IIT Bombay}} \hfill {{{\em {Apr '18 - Present}}}}
\begin{itemize}
\setlength\itemsep{0.01em}
\vspace{-0.2cm}
\item Conceptualized and organized 7 Multicity Competitions pan India thus increasing outreach
\vspace{-1.2mm}
\item Revamped the governing rules and regulations of 50+ competitions in various cultural genres
\end{itemize}
\vspace{-5mm}

\begin{flushleft}
\bf{\LARGE{T}\Large{ECHINAL} \LARGE{S}\Large{KILLS}}
\end{flushleft}
\vspace{-7mm}
\hrulefill
\vspace{-3mm}
\begin{itemize}
    \setlength\itemsep{0.01em}
    \item \textbf{Programming:} C/C++, Python, Java, R, MATLAB, Django, Bash, Kivy, HTML5, CSS3
    \vspace{-1.2mm}
    \item \textbf{Libraries:} PyTorch, Keras, MLpack, Tensorflow, Sklearn, OpenCV, Numpy, Scipy, Pandas
    \vspace{-1.2mm}
    \item \textbf{Software:}  Git, Github, Android Studio, \LaTeX, Adobe Photoshop, GNUplot, Jupyter
\end{itemize}

\vspace{-4mm}

\begin{flushleft}
\bf{\LARGE{K}\Large{EY} \LARGE{C}\Large{OURSES} \LARGE{U}\Large{NDERTAKEN}}
\end{flushleft}
\vspace{-7mm}
\hrulefill
\vspace{-3mm}
\begin{itemize}
\setlength\itemsep{0.01em}
\item \textbf{Electrical:} \nohyphens{Electronic Devices \& Circuits, Introduction to Electrical Systems, Network Theory, Microelectronics, Signals \& Systems**, Digital Systems**, Analog Systems**, Electrical machines \& Power Electronics**}
\vspace{-1.2mm}
\item \textbf{Mathematics and Statistics:} \nohyphens{Calculus, Linear Algebra, Differential Equtions 1 \& 2, Complex Analysis, Data Interpretation \& Analysis}
\vspace{-1.2mm}
\item \textbf{Other Courses:} \nohyphens{Quantum Physics \& Applications, Physical Chemistry, Economics, Computer Programming \& Utilization, Biology, Basis of Electricity \& Magnetism}
\end{itemize}
\vspace{-4mm}
\hfill {{{\em{(** to be completed by May '19)}}}} {{{\em{(* to be completed by Nov '18)}}}}
\hfill 
\begin{flushleft}
\bf{\LARGE{E}\Large{XTRA-}\LARGE{C}\Large{URRICULAR} \LARGE{A}\Large{CTIVIES}}
\end{flushleft}
\vspace{-6mm}
\hrulefill
\vspace{-1mm}
\begin{itemize}
    \vspace{-2mm}
    \item {\bf Active Player} in Indian {\bf Rainbow Six Siege}  Community
    \vspace{-3mm}
    \item Mentored a team for {\bf XLR8 2018 Robotics Competiton} who then won the {\bf 2\textsuperscript{nd} prize} 
    \vspace{-3mm}
    \item Active {\bf Quizzer} in many intra-school quiz competitions.
    \vspace{-3mm}
    \item Active delegate in debates and \textbf{Dynamic Speaker} at many public Speaking events in school
\end{itemize}
\end{document}
